\documentclass[11pt,a4paper]{article}
\usepackage[T1]{fontenc}
\usepackage[utf8]{inputenc}
\usepackage{lmodern}
\usepackage{amsmath,amssymb}
\usepackage{geometry}
\usepackage{hyperref}
\usepackage{enumitem}
\usepackage{graphicx}
\usepackage{microtype}
\geometry{margin=1in}
\setlist[itemize]{leftmargin=*, topsep=2pt, itemsep=2pt}
\title{Designing Molecularly Programmed Metasurfaces with Gaussian, Multiwfn, and S4}
\date{2025-11-09}
\begin{document}
\maketitle

\section{Overview}
Combining \textbf{Gaussian} (quantum chemistry), \textbf{Multiwfn} (molecular analysis), and \textbf{S4} (RCWA metasurface solver) enables \emph{molecularly programmed} metasurfaces. We outline (i) concrete project ideas, (ii) a physics-grounded workflow linking the tools, and (iii) a pipeline (with code) to convert TD-DFT results into dispersive $n(\lambda),\,k(\lambda)$ for S4.

\section{Potential Projects}
Using molecular resonances inside photonic metasurfaces yields useful functions:
\begin{itemize}
  \item \textbf{Molecularly Tuned Perfect Absorber (Critical Coupling):} Match a metasurface resonance to a dye's excitonic line so incident light is absorbed in the dye layer. Impact: ultrathin absorbers for sensing and pixels. TD-DFT $\to$ Clausius--Mossotti mixing $\to$ S4 \href{https://web.stanford.edu/group/fan/publication/Liu_ComputerPhysicsCommunications_183_2233_2012.pdf}{(S4)}.
  \item \textbf{All-Optical Switch with Photochromes:} Compute two states (cis/trans) to obtain two dispersions; design a metasurface at a high phase-sensitivity point to achieve large all-optical modulation.
  \item \textbf{Biosensor via Overlayer Index Shift:} Thin protein/DNA films shift narrow resonances; compute complex index from polarizability and predict detection limits in S4.
  \item \textbf{Polarization Control with Aligned Dyes:} Aligned dipoles yield anisotropic $\varepsilon$; simulate polarization-dependent phase in S4 with tensor permittivity.
\end{itemize}
\textit{Easy path:} The perfect absorber (first item) is experimentally straightforward: a periodic pattern plus a spin-coated dye film and passive R/T measurements.

\section{End-to-End Workflow}
\subsection*{A. Quantum Chemistry (Gaussian TD-DFT)}
Optimize the ground state; compute excited states and oscillator strengths with TD-DFT (range-separated hybrids such as CAM-B3LYP or $\omega$B97X-D recommended). Optional PCM captures environment shifts. Gaussian prints lines like: \texttt{Excited State 1: Singlet-A 2.35 eV 527.5 nm f=0.51} (use energies $E_j$ and strengths $f_j$).

\subsection*{B. Extract \& Verify (Multiwfn)}
Load LOG/fchk in Multiwfn's spectrum module to list $(E_j,f_j)$; sanity-check dominant visible transitions. See manual \href{https://sobereva.com/multiwfn/misc/Multiwfn_3.8_dev.pdf}{here}.

\subsection*{C. Molecules $\to$ Bulk $n,k$}
\textbf{Lorentz polarizability:}
\begin{equation}
\alpha_{\mathrm{mol}}(\omega)=\sum_j \frac{e^2}{m_e}\,\frac{f_j}{\omega_j^2-\omega^2-i\,\gamma_j\,\omega},\quad \omega_j=E_j/\hbar.
\end{equation}
\textbf{Clausius--Mossotti / Lorentz--Lorenz:}
\begin{equation}
\frac{\varepsilon_r-1}{\varepsilon_r+2}=\frac{N\,\alpha_{\mathrm{mol}}}{3\,\varepsilon_0}\;\Rightarrow\; \varepsilon_r=\frac{1+2X}{1-X},\; X=\frac{N\,\alpha_{\mathrm{mol}}}{3\,\varepsilon_0}.
\end{equation}
Include host by adding $X_{\mathrm{host}}=(n_{\mathrm{host}}^2-1)/(n_{\mathrm{host}}^2+2)$ to $X$. Then $\varepsilon_r=(n+ik)^2$.

\subsection*{D. Electromagnetics (S4 RCWA)}
Define air / patterned dielectric / dye film / substrate. For each wavelength, set dye $n(\lambda),k(\lambda)$, compute $R,T,A$, and sweep geometry to achieve critical coupling. See S4 paper \href{https://web.stanford.edu/group/fan/publication/Liu_ComputerPhysicsCommunications_183_2233_2012.pdf}{(Liu \& Fan, CPC 2012)}.

\section{TD-DFT $\to$ $n(\lambda),k(\lambda)$: Python}
\subsection*{Script (verbatim)}
\begin{verbatim}
#!/usr/bin/env python3
# gaussian_to_nk.py
# Convert Gaussian TD-DFT excited states (E_j, f_j) to dispersive n(lambda), k(lambda)
# via Lorentz oscillators + Clausius-Mossotti / Lorentz-Lorenz mixing.

import re, math, csv, argparse
from typing import List, Tuple

EPS0 = 8.8541878128e-12; QE = 1.602176634e-19; ME = 9.1093837015e-31
HBAR = 1.054571817e-34; C0 = 299792458.0; NA = 6.02214076e23; EV_TO_J = QE

EXCITED_STATE_RE = re.compile(r"Excited State\s+\d+:\s+.*?([\d\.]+)\s+eV\s+" \
                             r"[\d\.]+\s+nm\s+f=([\d\.]+)", re.I)

def parse_gaussian_td_log(path: str) -> List[Tuple[float, float]]:
    rows = []
    with open(path, 'r', errors='ignore') as fh:
        for line in fh:
            m = EXCITED_STATE_RE.search(line)
            if m: rows.append((float(m.group(1)), float(m.group(2))))
    if not rows: raise RuntimeError('No excited states found')
    return rows

def eV_to_omega(E_eV: float) -> float: return (E_eV*EV_TO_J)/HBAR
def eV_to_gamma(g_eV: float) -> float: return (g_eV*EV_TO_J)/HBAR

def lorentz_alpha_mol(omega: float, lines: List[Tuple[float,float]], gamma_eV: float) -> complex:
    alpha = 0+0j; gamma = eV_to_gamma(gamma_eV); pref = (QE**2)/ME
    for E_eV, f in lines:
        wj = eV_to_omega(E_eV)
        alpha += pref * f / ((wj**2 - omega**2) - 1j*gamma*omega)
    return alpha

def lorentz_lorenz_X(n: float) -> float:
    return (n*n - 1.0)/(n*n + 2.0)

def epsilon_from_X(X: complex) -> complex: return (1+2*X)/(1-X)

def nk_from_eps(eps_r: complex):
    nr = eps_r**0.5
    return (nr.real, abs(nr.imag))

def number_density(wt_frac: float, density_g_cm3: float, molar_mass_g_mol: float) -> float:
    rho = density_g_cm3*1000.0; M = molar_mass_g_mol/1000.0
    return ((wt_frac*rho)/M)*NA

def build_dispersion(lines, gamma_eV, N, lambdas_nm, host_n=None):
    X_host = lorentz_lorenz_X(host_n) if host_n is not None else 0.0
    out = []
    for lam_nm in lambdas_nm:
        omega = 2*math.pi*C0/(lam_nm*1e-9)
        alpha = lorentz_alpha_mol(omega, lines, gamma_eV)
        X = X_host + (N*alpha)/(3.0*EPS0)
        eps_r = epsilon_from_X(X)
        n,k = nk_from_eps(eps_r)
        out.append((lam_nm, n, k))
    return out

def main():
    ap = argparse.ArgumentParser(description='TD-DFT to n,k via Lorentz-Lorenz')
    ap.add_argument('--log', required=True)
    ap.add_argument('--molar_mass', type=float, required=True)
    ap.add_argument('--wt_percent', type=float, default=1.0)
    ap.add_argument('--density', type=float, default=1.2)
    ap.add_argument('--gamma', type=float, default=0.10)
    ap.add_argument('--host_n', type=float, default=None)
    ap.add_argument('--lambda_min', type=float, default=400.0)
    ap.add_argument('--lambda_max', type=float, default=800.0)
    ap.add_argument('--lambda_step', type=float, default=1.0)
    ap.add_argument('--out', default='nk.csv')
    args = ap.parse_args()

    lines = parse_gaussian_td_log(args.log)
    N = number_density(args.wt_percent/100.0, args.density, args.molar_mass)
    lambdas = [args.lambda_min + i*args.lambda_step
               for i in range(int((args.lambda_max-args.lambda_min)/args.lambda_step)+1)]
    table = build_dispersion(lines, args.gamma, N, lambdas, host_n=args.host_n)
    with open(args.out, 'w', newline='') as fh:
        w = csv.writer(fh); w.writerow(['lambda_nm','n','k']); w.writerows(table)
    if max(x[2] for x in table) < 1e-4:
        print('Note: k very small; increase wt% or gamma or check f_j')
    if args.wt_percent > 20.0:
        print('Warning: wt% > 20% may be unphysical; mixing may break down')

if __name__ == '__main__':
    main()
\end{verbatim}

\section{Using S4}
\begin{enumerate}
  \item Define materials/layers: air / grating (e.g., TiO$_2$ or Si$_3$N$_4$) / dye film / substrate (glass).
  \item For each $\lambda$, set dye $n,k$ from CSV; in Lua, update material, set frequency (e.g., \texttt{S:SetFrequency(c0/lambda)}), compute $R,T$.
  \item Sweep period, height, and film thickness to reach \emph{critical coupling} ($R\approx 0$, $T\approx 0$ at $\lambda_0$).
  \item Inspect fields; ensure energy concentrates in the dye at resonance.
\end{enumerate}

\paragraph{Tips} Converge Fourier orders; choose realistic $\gamma$ (50--120 meV typical); respect dye solubility; host index (\~1.5) helps mode overlap; validate with a uniform film before patterning.

\section{Example Next Steps}
Pick a strong visible dye (peak $\sim$600 nm); generate $n,k$ at 1 wt\% in PMMA (1.2 g/cm$^3$); design a 1D TiO$_2$ grating (period $\sim \lambda/n_{\mathrm{eff}}$) with a 50 nm dye film; optimize for $A=1-R-T$ peak near 600 nm; then fabricate (nanoimprint or commercial grating) and measure R/T.

\section*{References}
S4 RCWA solver: \href{https://web.stanford.edu/group/fan/publication/Liu_ComputerPhysicsCommunications_183_2233_2012.pdf}{Liu \& Fan, CPC 183, 2233 (2012)}. Gaussian TD-DFT outputs: \href{https://joaquinbarroso.com/2022/09/06/population-analysis-in-the-excited-state-with-gaussian/}{guide}. Lorentz--Lorenz relation: \href{https://www.theochem.ru.nl/~pwormer/Knowino/knowino.org/wiki/Lorentz-Lorenz_relation.html}{overview}. TD-DFT oscillator strength benchmarking: \href{https://pfloos.github.io/WEB_LOOS/pub/97.pdf}{study}.

\end{document}

