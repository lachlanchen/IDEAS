\documentclass[conference]{IEEEtran}
\usepackage[T1]{fontenc}
\usepackage[utf8]{inputenc}
\usepackage{lmodern}
\usepackage{amsmath,amssymb}
\usepackage{graphicx}
\usepackage{hyperref}
\usepackage{url}
\usepackage{verbatim}
\title{DORAEMON: Detached On-demand Rapid Absorption Energy Mechanism for Optimized Networks\\\large Strengthened Theory, Derivations, and Code-Ready Models}
\author{L. Chan, A. Chan, T. Nakamura}
\date{}
\begin{document}
\maketitle

\begin{abstract}
We develop a rigorous, code-verifiable framework for the DORAEMON tap-and-charge system. We derive closed-form expressions for resonant inductive coupling efficiency, supercapacitor absorption efficiency, and optimal control strategies for both the ultrafast charge (50--200 ms) and the long discharge phase into a battery. The paper consolidates the original IEEE manuscript (included verbatim as an appendix) and a strengthened theoretical treatment suitable for immediate simulation in MATLAB/Simulink or Python.
\end{abstract}

\section{Concept Overview}
DORAEMON decouples brief high-power wireless energy absorption from longer battery charging by buffering energy in a supercapacitor bank (``energy cavity''). A resonant inductive link at ISM frequency (e.g., 6.78\,MHz) delivers a pulse (50--200\,ms), then the device detaches and the supercapacitor discharges into the battery over minutes.

\section{Electromagnetic Coupling}
Let $L_1,L_2$ be the coil inductances, $R_1,R_2$ their series losses, and $M=k\sqrt{L_1L_2}$ the mutual inductance. With both tanks tuned to $\omega_0$, the maximum link efficiency under optimal load is
\begin{equation}
  \eta_{\max} = \frac{k^2 Q_1 Q_2}{\big(1+\sqrt{1+k^2 Q_1 Q_2}\big)^2},\quad Q_i = \frac{\omega_0 L_i}{R_i}.
\end{equation}
The optimal (reflected) load scales as $R_{L,\mathrm{opt}}\propto R_2\sqrt{1+k^2 Q_1Q_2}$. The link rise-time is governed by $\tau\sim \tfrac{2Q_1Q_2}{Q_1+Q_2}\,\omega_0^{-1}$, typically $\mathcal{O}(\mu\text{s})$ at MHz and $Q\sim10^2$, ensuring steady transfer throughout a 50\,ms contact.

\section{Rapid Energy Absorption}
Let the effective capacitance be $C_{\!\mathrm{eff}}(V_c)$ and ESR be $R_{\!\mathrm{ESR}}$. The charge dynamics are
\begin{equation}
  \frac{dV_c}{dt} = \frac{I_{\mathrm{in}}-I_{\mathrm{leak}}(V_c)}{C_{\!\mathrm{eff}}(V_c)},\qquad
  E(V_c)=\int_0^{V_c}\! C_{\!\mathrm{eff}}(V)\,V\,dV.
\end{equation}
Instantaneous storage efficiency (leakage negligible over 50\,ms) is
\begin{equation}
  \eta_{\mathrm{stor}}(I) = \frac{V_c I}{V_c I + I^2 R_{\!\mathrm{ESR}}} = \frac{1}{1+\tfrac{I R_{\!\mathrm{ESR}}}{V_c}},
\end{equation}
implying tapered current profiles (lower $I$ when $V_c$ is small) are superior to flat high current pulses. An optimal control over $t\in[0,T_c]$ maximizes $E(V_c(T_c))$ subject to $0\le I(t)\le I_{\max}$ and voltage/thermal limits. Numerically, bang--bang or ramp-up profiles emerge depending on constraints.

\section{Battery Discharge Optimization}
Using a Randles model with OCV$(\mathrm{SOC})$, $R_{\mathrm{int}}$, and one RC branch, a practical policy is CC--CV constrained by temperature:
\begin{equation}
  V_{\!\mathrm{bat}}=\mathrm{OCV}(\mathrm{SOC})+ I_{\!\mathrm{bat}} R_{\!\mathrm{int}} + V_{RC},\quad
  \tau_{RC} \dot V_{RC} + V_{RC} = I_{\!\mathrm{bat}} R_{RC}.
\end{equation}
Pontryagin-based fast-charge formulations typically yield ``max-current until a constraint, then ride the constraint''. This is readily implemented with MPC or simple clamps.

\section{Code-Ready Validation}
Each block admits compact simulation recipes in MATLAB/Simulink or Python: (i) mutual $k$ sweep vs alignment/separation; (ii) ODE integration for $V_c(t)$ with ESR loss and tapered $I(t)$; (iii) battery CC--CV with thermal clamp; (iv) full-link co-simulation (mutual inductor + rectifier + DC/DC).

\section*{Appendix: Original IEEE Manuscript (verbatim)}
\onecolumn
\begin{verbatim}
\documentclass[conference]{IEEEtran} \IEEEoverridecommandlockouts \usepackage{cite} \usepackage{amsmath,amssymb,amsfonts} \usepackage{algorithmic} \usepackage{graphicx} \usepackage{textcomp} \usepackage{xcolor} \usepackage{algorithm} \usepackage{algpseudocode} \usepackage{array} \usepackage{multirow} \def\BibTeX{{\rm B\kern-.05em{\sc i\kern-.025em b}\kern-.08em T\kern-.1667em\lower.7ex\hbox{E}\kern-.125emX}} \begin{document} \title{DORAEMON: Detached On-demand Rapid Absorption Energy Mechanism for Optimized Networks\\ {\footnotesize \textnormal{Mathematical Framework, Simulation, and Implementation of Energy Cavity Technology}}} \author{\IEEEauthorblockN{1\textsuperscript{st} Ayano Chan} \IEEEauthorblockA{\textit{Advanced Energy Systems Division} \\ \textit{The Art of Lazying}\\ Tokyo, Japan \\ ayano@lazying.art} \and \IEEEauthorblockN{2\textsuperscript{nd} Lachlan Chan} \IEEEauthorblockA{\textit{Computational Physics Department} \\ \textit{The Art of Lazying}\\ Hongkong, China \\ lach@lazying.art} \and \IEEEauthorblockN{3\textsuperscript{rd} Takeshi Nakamura} \IEEEauthorblockA{\textit{Institute for Future Technology} \\ \textit{Wireless Power Transfer Division}\\ Osaka, Japan \\ t.nakamura@ift.ac.jp} } \maketitle \begin{abstract} This paper presents DORAEMON (Detached On-demand Rapid Absorption Energy Mechanism for Optimized Networks), a revolutionary wireless charging paradigm with comprehensive mathematical framework and simulation validation. The system employs electromagnetic resonance theory, advanced energy storage dynamics, and optimization algorithms to achieve instantaneous power absorption exceeding 100W within 50ms contact time. We derive closed-form solutions for electromagnetic coupling efficiency, present rigorous energy storage differential equations, and develop novel optimization algorithms for discharge control. Comprehensive MATLAB/Simulink simulations validate theoretical predictions, demonstrating 94.7\% energy transfer efficiency and stable 30-minute post-detachment operation. The mathematical framework provides design guidelines for practical implementation across multiple power scales from IoT devices (mW) to electric vehicles (kW). \end{abstract} \begin{IEEEkeywords} wireless power transfer, electromagnetic theory, energy optimization, supercapacitor dynamics, numerical simulation, MATLAB \end{IEEEkeywords} \section{Introduction} The DORAEMON system addresses fundamental limitations in wireless power transfer through rigorous mathematical modeling and simulation-driven design. Current wireless charging systems require continuous electromagnetic coupling, limiting practical applications due to mobility constraints \cite{kurs2007wireless}. Our approach decouples energy absorption and storage phases through mathematically optimized electromagnetic resonance and electrochemical energy management. This work presents the complete mathematical framework underlying DORAEMON technology, including: \begin{enumerate} \item Rigorous electromagnetic field analysis with closed-form coupling solutions \item Comprehensive energy storage dynamics using differential equation systems \item Novel optimization algorithms for discharge control and efficiency maximization \item Detailed simulation methodology with MATLAB/Simulink implementation \item Experimental validation of theoretical predictions \end{enumerate} \section{Mathematical Framework} \subsection{Electromagnetic Field Analysis} \subsubsection{Maxwell Equations for Resonant Coupling} The electromagnetic fields in the DORAEMON system are governed by Maxwell's equations in the quasi-static approximation. For the resonant frequency $\omega_0 = 2\pi f_0$, the magnetic field distribution between transmitter and receiver coils is: \begin{equation} \vec{H}(\vec{r}, t) = \vec{H}_0(\vec{r}) e^{j(\omega_0 t + \phi)} \end{equation} where $\vec{H}_0(\vec{r})$ satisfies the vector wave equation: \begin{equation} \nabla^2 \vec{H}_0 + k_0^2 \epsilon_r \vec{H}_0 = 0 \end{equation} with $k_0 = \omega_0/c$ and relative permittivity $\epsilon_r$. \subsubsection{Coupled Circuit Analysis} The transmitter-receiver system forms a coupled oscillator described by: \begin{align} L_1 \frac{dI_1}{dt} + R_1 I_1 + \frac{1}{C_1}\int I_1 dt + M_{12} \frac{dI_2}{dt} &= V_{in}(t) \\ L_2 \frac{dI_2}{dt} + R_2 I_2 + \frac{1}{C_2}\int I_2 dt + M_{12} \frac{dI_1}{dt} &= 0 \end{align} where $M_{12} = k\sqrt{L_1 L_2}$ is the mutual inductance with coupling coefficient: \begin{equation} k = \frac{\mu_0}{4\pi} \oint_{C_1} \oint_{C_2} \frac{d\vec{l_1} \cdot d\vec{l_2}}{\vert\vec{r_1} - \vec{r_2}\vert} \end{equation} \subsubsection{Optimized Coupling Efficiency} Taking the Laplace transform and solving the coupled equations yields the power transfer efficiency: \begin{equation} \eta_{coupling} = \frac{k^2 Q_1 Q_2 R_L}{(R_1 + R_2)(R_L + R_2) + k^2 Q_1 Q_2 R_2} \cdot \frac{1}{1 + (\Delta\omega \tau)^2} \end{equation} where $Q_i = \omega_0 L_i / R_i$, $\Delta\omega$ is frequency detuning, and $\tau = 2Q_1Q_2/(Q_1 + Q_2)$ is the coupling time constant. The maximum efficiency occurs at optimal load resistance: \begin{equation} R_{L,opt} = R_2 \sqrt{1 + k^2 Q_1 Q_2} \end{equation} yielding maximum efficiency: \begin{equation} \eta_{max} = \frac{k^2 Q_1 Q_2}{(1 + \sqrt{1 + k^2 Q_1 Q_2})^2} \end{equation} \subsection{Energy Storage Dynamics} \subsubsection{Supercapacitor Charging Model} The supercapacitor bank charging during the absorption phase follows the nonlinear differential equation: \begin{equation} C_{eff}(V_c) \frac{dV_c}{dt} = I_{charge}(t) - I_{leakage}(V_c, T) \end{equation} where the effective capacitance varies with voltage: \begin{equation} C_{eff}(V_c) = C_0 \left(1 - \alpha \frac{V_c}{V_{rated}} + \beta \left(\frac{V_c}{V_{rated}}\right)^2\right) \end{equation} and leakage current depends on voltage and temperature: \begin{equation} I_{leakage}(V_c, T) = I_0 \exp\left(\frac{V_c - V_{th}}{nV_T}\right) \exp\left(\frac{E_a}{kT}\right) \end{equation} \subsubsection{Energy Balance Equation} The instantaneous stored energy is: \begin{equation} E_{stored}(t) = \int_0^{V_c(t)} C_{eff}(V) V dV \end{equation} The power balance during charging: \begin{equation} P_{input}(t) = P_{stored}(t) + P_{loss}(t) + P_{leakage}(t) \end{equation} where: \begin{align} P_{stored}(t) &= V_c(t) C_{eff}(V_c) \frac{dV_c}{dt} \\ P_{loss}(t) &= I_{charge}^2(t) R_{ESR}(f, T) \\ P_{leakage}(t) &= V_c(t) I_{leakage}(V_c, T) \end{align} \subsubsection{Optimal Charging Algorithm} To maximize energy storage efficiency, we solve the optimization problem: \begin{equation} \max_{I_{charge}(t)} \int_0^{T_{charge}} \eta_{storage}(I_{charge}(t)) P_{input}(t) dt \end{equation} subject to: \begin{align} 0 \leq I_{charge}(t) &\leq I_{max} \\ V_c(t) &\leq V_{max} \\ T_{junction}(t) &\leq T_{max} \end{align} The storage efficiency is: \begin{equation} \eta_{storage}(I) = \frac{P_{stored}}{P_{input}} = \frac{1}{1 + \frac{I R_{ESR}}{V_c} + \frac{I_{leakage}}{I}} \end{equation} \subsection{Discharge Optimization Theory} \subsubsection{Battery Charging Dynamics} The battery charging process is modeled using the Randles equivalent circuit: \begin{equation} V_{battery}(t) = V_{OCV}(SOC) + I_{discharge}(t) R_{internal} + V_{RC}(t) \end{equation} where $V_{RC}(t)$ satisfies: \begin{equation} \tau_{RC} \frac{dV_{RC}}{dt} + V_{RC} = I_{discharge}(t) R_{RC} \end{equation} with time constant $\tau_{RC} = R_{RC} C_{RC}$. \subsubsection{Optimal Discharge Control} The optimal discharge current profile minimizes charging time while maximizing battery health: \begin{equation} \min_{I_d(t)} J = \int_0^{T_d} \left[\alpha_1 (I_d(t) - I_{opt}(SOC))^2 + \alpha_2 T_{bat}^2(t) + \alpha_3 \right] dt \end{equation} subject to energy conservation: \begin{equation} \int_0^{T_d} I_d(t) dt = Q_{target} \end{equation} and thermal constraint: \begin{equation} T_{bat}(t) = T_{ambient} + R_{thermal} I_d^2(t) R_{internal} \leq T_{max} \end{equation} Using Pontryagin's maximum principle, the optimal control is: \begin{equation} I_d^*(t) = \frac{\lambda(t) - 2\alpha_1 I_{opt}(SOC(t))}{2(\alpha_1 + \alpha_2 R_{thermal} R_{internal})} \end{equation} where $\lambda(t)$ is the co-state variable satisfying: \begin{equation} \frac{d\lambda}{dt} = 2\alpha_1 \frac{dI_{opt}}{dSOC} \frac{dSOC}{dt} \end{equation} \section{Simulation Methodology and Implementation} \subsection{MATLAB/Simulink Framework} The complete DORAEMON system simulation was implemented in MATLAB R2023b with Simulink for real-time modeling. The simulation architecture consists of four main modules: \begin{enumerate} \item Electromagnetic coupling solver using finite element method (FEM) \item Supercapacitor dynamics simulator with thermal modeling \item Battery charging model with electrochemical dynamics \item System optimization and control algorithms \end{enumerate} \subsection{Electromagnetic Simulation Algorithm} \begin{algorithm} \caption{FEM Electromagnetic Coupling Solver} \begin{algorithmic}[1] \Procedure{SolveEMCoupling}{$geometry, frequency, materials$} \State Initialize mesh with adaptive refinement \State $\mathbf{K} \leftarrow$ AssembleStiffnessMatrix($geometry, materials$) \State $\mathbf{M} \leftarrow$ AssembleMassMatrix($geometry, materials$) \State $\mathbf{A} \leftarrow \mathbf{K} - \omega^2 \mathbf{M}$ \State $\mathbf{b} \leftarrow$ AssembleSourceVector($excitation$) \State $\mathbf{H} \leftarrow$ SolveLU($\mathbf{A}, $\mathbf{b}$) \State $k_{coupling} \leftarrow$ CalculateCouplingCoeff($\mathbf{H}$) \State $\eta_{max} \leftarrow$ ComputeMaxEfficiency($k_{coupling}, Q_1, Q_2$) \Return $k_{coupling}, \eta_{max}, \mathbf{H}$ \EndProcedure \end{algorithmic} \end{algorithm} \subsection{Energy Storage Simulation} \begin{algorithm} \caption{Supercapacitor Dynamics Solver} \begin{algorithmic}[1] \Procedure{SimulateSupercapacitor}{$I_{input}(t), T_{ambient}, \Delta t$} \State Initialize: $V_c(0) = 0, T(0) = T_{ambient}$ \For{$t = 0$ to $T_{total}$ step $\Delta t$} \State $C_{eff} \leftarrow C_0(1 - \alpha V_c/V_{rated} + \beta (V_c/V_{rated})^2)$ \State $I_{leakage} \leftarrow I_0 \exp((V_c - V_{th})/(nV_T)) \exp(E_a/(kT))$ \State $R_{ESR} \leftarrow R_0(1 + \alpha_R (T - T_0))$ \State $P_{loss} \leftarrow I_{input}^2 \cdot R_{ESR}$ \State $\frac{dT}{dt} \leftarrow (P_{loss} - (T - T_{ambient})/R_{thermal})/C_{thermal}$ \State $\frac{dV_c}{dt} \leftarrow (I_{input} - I_{leakage})/C_{eff}$ \State $V_c \leftarrow V_c + \frac{dV_c}{dt} \cdot \Delta t$ \State $T \leftarrow T + \frac{dT}{dt} \cdot \Delta t$ \State Store: $V_c(t), T(t), E_{stored}(t)$ \EndFor \Return $V_c(t), T(t), E_{stored}(t)$ \EndProcedure \end{algorithmic} \end{algorithm} \subsection{Optimal Control Implementation} \begin{algorithm} \caption{Optimal Discharge Control Algorithm} \begin{algorithmic}[1] \Procedure{OptimalDischarge}{$E_{initial}, Q_{target}, T_{max}$} \State Initialize: $SOC(0), \lambda(0), I_d(0)$ \State $N \leftarrow T_{discharge}/\Delta t$ \For{$k = 1$ to $N$} \State $I_{opt} \leftarrow$ GetOptimalCurrent($SOC(k), T_{bat}(k)$) \State $\lambda(k+1) \leftarrow \lambda(k) + 2\alpha_1 \frac{dI_{opt}}{dSOC} \frac{dSOC}{dt} \Delta t$ \State $I_d^*(k) \leftarrow \frac{\lambda(k) - 2\alpha_1 I_{opt}}{2(\alpha_1 + \alpha_2 R_{th} R_{int})}$ \State Clamp: $I_d^*(k) \leftarrow \min(\max(I_d^*(k), 0), I_{max})$ \State $V_{bat}(k+1) \leftarrow V_{OCV}(SOC(k)) + I_d^*(k) R_{int} + V_{RC}(k)$ \State $SOC(k+1) \leftarrow SOC(k) + I_d^*(k) \Delta t / Q_{rated}$ \State $T_{bat}(k+1) \leftarrow T_{amb} + R_{th} (I_d^*(k))^2 R_{int}$ \If{$T_{bat}(k+1) > T_{max}$} \State $I_d^*(k) \leftarrow \sqrt{(T_{max} - T_{amb})/(R_{th} R_{int})}$ \EndIf \EndFor \Return $I_d^*(t), V_{bat}(t), SOC(t)$ \EndProcedure \end{algorithmic} \end{algorithm} \subsection{Software Architecture} The simulation environment utilizes: \begin{itemize} \item \textbf{COMSOL Multiphysics 6.1}: Electromagnetic field analysis and mesh generation \item \textbf{MATLAB R2023b}: Main simulation engine and optimization algorithms \item \textbf{Simulink}: Real-time system modeling and control design \item \textbf{Optimization Toolbox}: Constrained optimization solvers (fmincon, ga) \item \textbf{Control System Toolbox}: Model predictive control implementation \item \textbf{Parallel Computing Toolbox}: Monte Carlo simulations and parameter sweeps \end{itemize} \subsection{Validation Methodology} The simulation accuracy is validated through: \begin{enumerate} \item Comparison with analytical solutions for simplified geometries \item Experimental validation using prototype measurements \item Cross-validation with independent simulation tools (ANSYS Maxwell) \item Convergence studies for mesh refinement and time step selection \end{enumerate} \section{Simulation Results and Analysis} \subsection{Electromagnetic Coupling Optimization} Parametric sweeps across frequency (1-10 MHz), coil separation (5-50 mm), and quality factors (50-500) revealed optimal operating conditions: \begin{itemize} \item Resonant frequency: $f_0 = 6.78$ MHz (ISM band) \item Optimal separation: $d_{opt} = 12$ mm \item Quality factors: $Q_1 = 200, Q_2 = 150$ \item Maximum coupling coefficient: $k_{max} = 0.42$ \item Peak efficiency: $\eta_{max} = 94.7\%$ \end{itemize} \subsection{Energy Storage Performance} Monte Carlo simulations (10,000 iterations) with parameter variations showed: \begin{align} \text{Energy density} &= 12.3 \pm 0.8 \text{ Wh/kg} \\ \text{Power density} &= 8.2 \pm 0.5 \text{ kW/kg} \\ \text{Charge time (0-95\%)} &= 47 \pm 3 \text{ ms} \\ \text{Efficiency} &= 92.1 \pm 1.2 \% \end{align} \subsection{Thermal Analysis} Thermal simulations revealed temperature distributions during high-power charging: \begin{equation} T_{max} = T_{ambient} + \frac{P_{loss} \cdot R_{thermal}}{1 + \tau_{thermal} s} \end{equation} Maximum component temperature remained below 45°C for ambient temperatures up to 35°C. \subsection{System Integration Results} Complete system simulations demonstrated: \begin{itemize} \item Contact time requirement: 50-200 ms for full energy transfer \item Post-detachment operation: 30-45 minutes stable charging \item Overall system efficiency: 87.3\% (including conversion losses) \item Scalability range: 1 mW to 100 kW power levels \end{itemize} \section{Experimental Validation} \subsection{Prototype Specifications} A laboratory prototype was constructed with the following specifications: \begin{table}[htbp] \caption{DORAEMON Prototype Specifications} \begin{center} \begin{tabular}{|l|c|} \hline \textbf{Parameter} & \textbf{Value} \\ \hline Transmitter Power & 200 W \\ Operating Frequency & 6.78 MHz \\ Supercapacitor Bank & 50 F, 16 V \\ Energy Storage & 6.4 kJ \\ Peak Charging Current & 100 A \\ Contact Time & 50-500 ms \\ Efficiency (measured) & 91.8\% \\ \hline \end{tabular} \end{center} \end{table} \subsection{Measurement Results} Experimental validation confirmed simulation predictions within 3\% accuracy: \begin{align} \eta_{experimental} &= 91.8\% \quad (\eta_{simulation} = 94.7\%) \\ P_{max,exp} &= 98.2 \text{ W} \quad (P_{max,sim} = 100 \text{ W}) \\ t_{charge,exp} &= 52 \text{ ms} \quad (t_{charge,sim} = 47 \text{ ms}) \end{align} \section{Performance Comparison} \begin{table}[htbp] \caption{Comprehensive Performance Comparison} \begin{center} \begin{tabular}{|l|c|c|c|c|} \hline \textbf{Metric} & \textbf{Qi 1.3} & \textbf{AirFuel} & \textbf{WiTricity} & \textbf{DORAEMON} \\ \hline Max Power (W) & 15 & 50 & 3300 & 100 \\ \hline Efficiency (\%) & 70 & 75 & 85 & 94.7 \\ \hline Range (mm) & 5 & 15 & 200 & 12* \\ \hline Contact Time & Continuous & Continuous & Continuous & 50 ms \\ \hline Mobility & None & Limited & Limited & Full \\ \hline Frequency (MHz) & 0.125 & 6.78 & 6.78 & 6.78 \\ \hline \multicolumn{5}{l}{*During energy absorption phase only} \end{tabular} \end{center} \end{table} \section{Applications and Future Directions} \subsection{Scalability Analysis} The mathematical framework enables scaling across power levels: \begin{align} P_{scaled} &= P_{base} \left(\frac{V_{scaled}}{V_{base}}\right)^2 \left(\frac{f_{scaled}}{f_{base}}\right) \\ \eta_{scaled} &= \eta_{base} \sqrt{\frac{Q_{scaled}}{Q_{base}}} \end{align} \subsection{Advanced Control Strategies} Future implementations will incorporate: \begin{itemize} \item Machine learning-based efficiency optimization \item Adaptive impedance matching using varactor arrays \item Multi-objective optimization for simultaneous efficiency and speed \item Predictive thermal management with model predictive control \end{itemize} \subsection{Commercial Viability} Economic analysis indicates break-even at production volumes exceeding 100,000 units, with projected cost reduction of 60\% compared to current wireless charging solutions at scale. \section{Conclusion} The DORAEMON system represents a fundamental advancement in wireless power transfer, validated through comprehensive mathematical modeling and simulation. The rigorous electromagnetic analysis, energy storage optimization, and control algorithms provide a complete framework for practical implementation. Key achievements include: \begin{enumerate} \item Theoretical maximum efficiency of 94.7\% with experimental validation at 91.8\% \item Rapid energy absorption in 50 ms enabling true mobile wireless charging \item Scalable design framework spanning six orders of magnitude in power \item Complete simulation environment for future development and optimization \end{enumerate} The mathematical rigor and simulation validation demonstrate the feasibility of energy cavity technology for next-generation wireless power systems, opening new possibilities for ubiquitous wireless charging across consumer electronics, IoT devices, and electric vehicles. \section*{Acknowledgment} The authors acknowledge The Art of Lazying for computational resources and experimental facilities. Special thanks to Dr. Hiroshi Yamamoto for electromagnetic field theory guidance and the MATLAB development team for simulation platform support. \begin{thebibliography}{00} \bibitem{kurs2007wireless} A. Kurs, A. Karalis, R. Moffatt, J. D. Joannopoulos, P. Fisher, and M. Soljačić, "Wireless power transfer via strongly coupled magnetic resonances," \textit{Science}, vol. 317, no. 5834, pp. 83-86, 2007. \bibitem{li2015wireless} S. Li and C. C. Mi, "Wireless power transfer for electric vehicle applications," \textit{IEEE Journal of Emerging and Selected Topics in Power Electronics}, vol. 3, no. 1, pp. 4-17, 2015. \bibitem{tesla1914wireless} N. Tesla, "Apparatus for transmitting electrical energy," U.S. Patent 1,119,732, Dec. 1, 1914. \bibitem{sample2011analysis} A. P. Sample, D. A. Meyer, and J. R. Smith, "Analysis, experimental results, and range adaptation of magnetically coupled resonators for wireless power transfer," \textit{IEEE Transactions on Industrial Electronics}, vol. 58, no. 2, pp. 544-554, 2011. \bibitem{hui2014planar} S. Y. R. Hui, W. Zhong, and C. K. Lee, "A critical review of recent progress in mid-range wireless power transfer," \textit{IEEE Transactions on Power Electronics}, vol. 29, no. 9, pp. 4500-4511, 2014. \bibitem{karalis2008efficient} A. Karalis, J. D. Joannopoulos, and M. Soljačić, "Efficient wireless non-radiative mid-range energy transfer," \textit{Annals of Physics}, vol. 323, no. 1, pp. 34-48, 2008. \bibitem{zhang2019wireless} Z. Zhang, H. Pang, A. Georgiadis, and C. Cecati, "Wireless power transfer—an overview," \textit{IEEE Transactions on Industrial Electronics}, vol. 66, no. 2, pp. 1044-1058, 2019. \bibitem{pontryagin1962mathematical} L. S. Pontryagin, V. G. Boltyanskii, R. V. Gamkrelidze, and E. F. Mishchenko, \textit{The Mathematical Theory of Optimal Processes}. New York: Wiley, 1962. \bibitem{conway2006electrochemical} B. E. Conway, \textit{Electrochemical Supercapacitors: Scientific Fundamentals and Technological Applications}. New York: Plenum Press, 1999. \bibitem{matlab2023} \textit{MATLAB and Simulink Documentation}, The MathWorks Inc., Natick, MA, 2023. \end{thebibliography} \end{document}
\end{verbatim}
\twocolumn

\end{document}

