\documentclass[11pt]{ctexart}
\usepackage{geometry}
\geometry{margin=1in}
\usepackage{hyperref}
\usepackage{amsmath,amssymb}
\usepackage{microtype}
% 固定日文/中文字体,避免豆腐
\setCJKmainfont[
  BoldFont={Noto Serif CJK JP SemiBold}
]{Noto Serif CJK JP}
\setCJKsansfont[
  BoldFont={Noto Sans CJK JP Medium}
]{Noto Sans CJK JP}
\setCJKmonofont{Noto Sans Mono CJK JP}
\XeTeXlinebreaklocale "zh"
\XeTeXlinebreakskip = 0pt plus 1pt

\title{用通俗中文讲清楚:DORAEMON「一碰即充」完整说明}
\author{LazyingArt}
\date{}

\begin{document}
\maketitle

\begin{quote}
目标:先讲\textbf{直觉},再讲\textbf{问题—方案—为什么能成—方法细节—影响与应用},最后给一个\textbf{小算例}和\textbf{如何用代码做验证}的思路。整篇可直接当说明文/提案文使用。
\end{quote}

\section{1) 直觉:像“磁性插头 + 海绵”}
把手机/设备轻轻一“贴”,\textbf{两只线圈}就像“磁性插头”一样对上了频率(\textbf{共振}),能量顺着\textbf{近场磁场}涌过来;设备里有一块\textbf{能量海绵}(\textbf{超级电容}),\textbf{几十毫秒内瞬间吸能},然后在接下来的几分钟里\textbf{慢慢把能量喂给电池}。

关键点:这不是要在 50~ms 内把\textbf{电池}充满,而是先把能量塞进“海绵”(超级电容),再\textbf{温和}地给电池。——\textbf{“快”与“稳”被时间上拆分}。

\section{2) 传统无线充电的痛点}
\begin{enumerate}
  \item \textbf{必须一直对准}:放偏就掉速、掉效。
  \item \textbf{速度慢}:功率不高,要放很久。
  \item \textbf{移动不友好}:边走边充基本不现实。
  \item \textbf{电池不耐“野蛮快充”}:化学反应有上限,大电流易发热、掉寿命。
\end{enumerate}

\section{3) 为什么 DORAEMON 能工作(核心直觉)}
\begin{itemize}
  \item \textbf{共振放大}:两线圈调到同一频率(如 6.78~MHz),近场磁耦合像“跷跷板”一样高效传能;\textbf{对得准}时,耦合效率可很高。
  \item \textbf{能量缓冲}:超级电容(功率密度高)能在\textbf{几十毫秒吸收大电流},这点电池做不到;相当于先把能量“存到小水缸”。
  \item \textbf{时间尺度解耦}:\textbf{50~ms}完成“吸能”,\textbf{30~分钟}完成“喂电池”。电池看到的是\textbf{温和、可控}的电流,更护寿命。
  \item \textbf{智能控流降损}:电容电压低时若“猛灌”,$I^2R$ 热损耗大。\textbf{先小—再大—后小}的电流波形显著提高“吸能效率”。
\end{itemize}

\section{4) 系统全貌(组成与分工)}
\textbf{发射端}
\begin{itemize}
  \item 高频驱动(功放)
  \item 发射线圈
  \item 调谐/匹配网络(可变)
\end{itemize}
\textbf{接收端}
\begin{itemize}
  \item 接收线圈
  \item 低损整流(优先同步整流)
  \item \textbf{超级电容(能量腔)}
  \item DC-DC 转换器(给电池/负载)
  \item 保护与测量(电流/电压/温度)
\end{itemize}
\textbf{感知与控制}
\begin{itemize}
  \item 合法设备与对准检测 / 异物检测
  \item 频率/相位/占空比调节、自动匹配
  \item 电流限幅、过温/过压/欠压保护
  \item 充电策略(入腔侧、给电池侧)
\end{itemize}

\section{5) 方法细节:一次“贴靠”的全过程(0 \texorpdfstring{$\to$}{->} 30~分钟)}
\subsection*{A.~0--50~ms:瞬时吸能(“海绵进水”)}
\begin{enumerate}
  \item \textbf{识别与对中}:确认合法设备、线圈基本对中(磁吸/定位结构辅助)。
  \item \textbf{建立共振}:两侧频率一致,近场磁耦合“通车”。
  \item \textbf{整流入腔}:接收侧把高频交流整流成直流,喂给超级电容。
  \item \textbf{电流波形要点:}
    \begin{itemize}
      \item \textbf{起步}(电容电压低):\textbf{中等电流},避免 $I^2R$ 占比过大;
      \item \textbf{中段}(电压升高、效率更好):逐步\textbf{拉高电流},快速装能量;
      \item \textbf{结束前}:逼近电容安全电压或时间将尽,\textbf{平滑回落}防过冲。
    \end{itemize}
  \item \textbf{温控与保护}:实时监温,必要时限流/停机;异物(硬币等)检测到立刻断能。
\end{enumerate}

\begin{quote}
\textbf{直觉总结}:电容电压越高,同样电流带来的\textbf{有效存能占比越大}。先把“底”垫起来再加速,整体更省损、效率更高。
\end{quote}

\subsection*{B.~50~ms--30~min:温和给电池(“海绵放水”)}
\begin{enumerate}
  \item \textbf{DC-DC 转换}:把电容较高电压变成电池所需电压/电流。
  \item \textbf{电池侧策略:}
  \begin{itemize}
    \item \textbf{类似 CC-CV}:前期用\textbf{恒定且安全的电流}(遵守电池 C-rate),电池电压接近上限时转为“电压受控”,\textbf{电流自然回落};
    \item \textbf{温度优先}:温升过高立即降档。
  \end{itemize}
  \item \textbf{合理的工作区}:不必把电容从 0~V 充到满,也不必榨到见底;让它\textbf{在高效率电压区间来回工作},平均效率高、发热低。
\end{enumerate}

\section{6) 一个小算例(把量级“算清楚”)}
\textbf{目标}:在 50~ms 内注入 5~J(等效平均功率 100~W),向一个等效 10~F 的超级电容“加一点电”。

\noindent 理想能量关系:
\[
  E=\tfrac{1}{2}\, C\, \Delta V^2
\]
代入:
\[
  5=\tfrac{1}{2}\times 10 \times \Delta V^2 \;\Rightarrow\; \Delta V=1\ \text{V}
\]
50~ms 内的平均电流:
\[
  I=\frac{C\,\Delta V}{\Delta t}= \frac{10\times 1}{0.05}=200\ \text{A}
\]
(峰值可由整流/滤波分担)

\noindent \textbf{ESR 决定损耗}:若等效 ESR = 0.2~m$\Omega$,
\[
  I^2R \,\Delta t = 200^2\times 0.0002\times 0.05 = 0.4\ \text{J}
\]
吸能效率 $\approx \tfrac{5}{5+0.4}=92.6\%$(很可观)。

若 ESR = 1~m$\Omega$,损耗变 2~J,效率掉到约 71\%。

\noindent \textbf{结论}:要么把 ESR 做到极低(并联多颗、优选器件、同步整流、减小寄生),要么不要从 0~V “猛灌”——这正是\textbf{三段式电流 + 合理电压工作区}的价值。

\begin{quote}
注:很多应用不需要 10~F;\textbf{几法拉}更容易在 50~ms 内形成可观 $\Delta V$,功率与损耗也更可控。
\end{quote}

\section{7) 关键挑战与边界条件}
\begin{itemize}
  \item \textbf{对准容差}:偏一点 $k$ 下降,效率/功率跟着掉;用\textbf{物理定位/磁吸} + 线圈形状优化提高容差。
  \item \textbf{电磁兼容与安全}:6.78~MHz 属 ISM 频段,但仍需\textbf{异物检测、屏蔽/导磁片}、场强管理。
  \item \textbf{峰值电流能力}:整流器件、母线、走线要承受 100--200~A 级短时冲击;优先\textbf{同步整流}、低寄生布局。
  \item \textbf{热管理}:虽是 50~ms,但峰值功率带来瞬时发热;需要\textbf{足够热容 + 散热路径},并限制“高频次连击”。
  \item \textbf{电池友好策略}:\textbf{温度与寿命优先},不要追求极限拉满。
\end{itemize}

\section{8) 影响与应用场景}
\begin{itemize}
  \item \textbf{用户体验}:手机/耳机/手表“一碰就走”,\textbf{几十毫秒换来几分钟到半小时续航},摆脱“趴在充电板上”。
  \item \textbf{IoT/工业}:巡检机器人、AGV、传感节点,路过时“抬手一碰”即可补能,\textbf{不停机}。
  \item \textbf{交通/物流}:对接“短停快充”(站台/工位),把等待压缩到\textbf{肌肉记忆级}。
  \item \textbf{电池寿命}:把“暴力快充”交给超级电容,电池只见“柔和电流”,\textbf{更耐用}。
\end{itemize}

\section{9) 这套方法为什么靠谱(一句话)}
\textbf{物理}有共振耦合保证短时\textbf{高效传能};\textbf{器件}有超级电容承受\textbf{巨瞬功率};\textbf{控制}用三段式电流/合适工作区把\textbf{损耗压低};\textbf{系统}把\textbf{快}与\textbf{稳}拆开,电池安全、体验简单。

\section{10) 如何用“代码做实验”(最小闭环思路)}
不急做硬件,先把“数字样机”在电脑里跑起来,评估量级与敏感性。

\subsection*{(1) 线圈耦合(近似)}
\begin{itemize}
  \item 用等效电源 + 负载模型,设置耦合效率 $\eta_{\text{coupling}}$(如 85--95\%),随\textbf{偏移/频偏/距离}变化。
  \item 输出:\textbf{50~ms 内可提供到整流端的能量(J)}。
\end{itemize}

\subsection*{(2) 超级电容充电 ODE}
\begin{itemize}
  \item 核心方程:
    \[
      \frac{dV_c}{dt}=\frac{I(t)-I_{\text{leak}}(V_c)}{C_{\text{eff}}(V_c)},\quad
      P_{\text{loss}}=I^2 R_{\text{ESR}}
    \]
    可加简单热模型(热容 + 热阻)。
  \item 试三类电流:\textbf{恒流}、\textbf{三段式}(先中后高再回落)、\textbf{优化搜索}(把 50~ms 切段,求每段电流)。
  \item 记录:\textbf{吸能效率、峰值温升、末端电压}。
\end{itemize}

\subsection*{(3) 电池侧放电}
\begin{itemize}
  \item 电池用简化 \textbf{Randles} 等效($V_{\text{OCV}}(\mathrm{SOC})$、$R_{\text{int}}$、一阶 RC)。
  \item DC-DC 设“\textbf{安全恒流上限 + 温度限流}”,再过渡到“电压受控”。
  \item 记录:\textbf{30~min 内 SOC 提升、温升、平均效率}。
\end{itemize}

\subsection*{(4) 端到端指标}
\[
  \eta_{\text{total}}=
  \eta_{\text{coupling}}\times
  \eta_{\text{rect}}\times
  \eta_{\text{storage}}\times
  \eta_{\text{dc-dc}}
\]
关注:\textbf{每次贴靠注入能量(J/次)}、\textbf{效率}、\textbf{热安全}、以及对\textbf{对准/ESR/频偏}的敏感性。

\section*{小结(要点速览)}
\begin{itemize}
  \item \textbf{问题}:想要“又快又自由”的无线充电,传统“长时间趴板”做不到。
  \item \textbf{直觉}:像磁性插头 + 海绵,\textbf{先快吸能、后慢喂电}。
  \item \textbf{方法}:\textbf{共振耦合}高效传能 + \textbf{超级电容}扛瞬时功率 + \textbf{三段式电流}与温控保护 + \textbf{CC-CV 给电池}。
  \item \textbf{为什么能成}:物理可行、器件能扛、控制能优化、系统把快与稳拆开。
  \item \textbf{应用}:极简体验、不断电补能、对电池更友好;先从小功率场景快速落地。
\end{itemize}

\section*{附:公式/名词小抄(便于查阅)}
\begin{itemize}
  \item 电容能量:$E=\tfrac{1}{2} C V^2$。
  \item 充电速率:$I=C\,\dfrac{\Delta V}{\Delta t}$。
  \item ESR 损耗:$E_{\text{loss}}=I^2 R_{\text{ESR}} \, \Delta t$。
  \item 吸能效率(近似):$\eta_{\text{storage}}=\dfrac{V_c I}{V_c I + I^2 R_{\text{ESR}}}=\dfrac{1}{1+\tfrac{I R_{\text{ESR}}}{V_c}}$。
  \item CC-CV:恒流(CC)到达电压阈值后转恒压(CV),电流自然回落。
  \item 6.78~MHz:常用 ISM 频段之一(近场共振耦合)。
\end{itemize}

\section*{想要进一步:下一步可做什么?}
我们可以把上述“数字样机”的\textbf{最简 Python/MATLAB 脚本}搭起来:线圈耦合近似 + 超级电容 ODE(含 ESR/热)+ 三段式电流生成器 + 电池侧 CC-CV 控制。一键跑完就能看到\textbf{每次贴靠注入多少焦耳}、\textbf{热是否安全}、以及\textbf{对齐偏差/ESR}对效率的影响。

\end{document}
